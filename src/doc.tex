% Article template for Mathematics Magazine
% Revised 7/2002  Thanks for Greg St. George
% \documentclass[12pt]{article}
\documentclass[sigconf]{acmart}
\usepackage{amssymb}
\usepackage{array}
\usepackage[ngerman]{babel}
\usepackage[utf8]{inputenc}
\usepackage{prftree}
\usepackage{amsmath}
% \usepackage{pf2}
\usepackage{graphicx}
\usepackage{tikz}
\usepackage{dot2texi}
\usetikzlibrary{shapes,arrows}
\renewcommand{\baselinestretch}{1.2}
%This is the command that spaces the manuscript for easy reading
\newtheorem{zeige}{Zeige}

\newtheorem{corollar}{Korollar}
\newtheorem{satz}{Satz}
% \newtheorem{lemma}{Lemma}
% \newtheorem{definition}{Definition}
\newtheorem{kollar}{Kollar}
\newtheorem{beispiel}{Beispiel}
% \newtheorem{theorem}{Theorem}
%todo
\usepackage[colorinlistoftodos,prependcaption,textsize=tiny]{todonotes}
\usepackage{xargs}                      % Use more than one optional parameter in a new commands
\newcommandx{\QUESTION}[2][1=]{\todo[linecolor=none,backgroundcolor=blue!15,bordercolor=none,#1]{\textbf{QUESTION: }#2}}
\newcommandx{\TODO}[2][1=]{\todo[inline,linecolor=none,backgroundcolor=blue!15,bordercolor=none,#1]{\textbf{TODO: }#2}}


\newenvironment {llstep}
  [0]
  {\[ \begin{array}{lcl}}
  {\end{array} \]}

\begin{document}

\title{Linear logic as a partly-executable specification language for smart contracts}
\author{Denis Erfurt \and Jack Ek}

\begin{abstract}
  asd asdas
\end{abstract}

\maketitle

\section{Introduction}
We propose to use a \emph{logical programming language} based on linear logic to specify smart contracts. For those unfamiliar with the logical programming paradigm, it means that the programmer uses logical propositions to specify \emph{what} the program should do or \emph{which constraints} it should satisfy, but not \emph{how} this is achieved. From this, the computer creates a suitable program and executes it. In short, in logical programming, the specification is executable and the program is correct-by-construction, although there are caveats to this interpretation which we will discuss later.

For those coming from a typed functional programming background, it can help to think of a logical language as a typed functional language, but in which the programmer only writes type signatures, no functions. This understanding is based directly on the Curry-Howard correspondence, which essentially states that type signatures are the same things as logical propositions, and programs satisfying type signatures are the same things as proofs of those propositions.

Traditionally, logical programming is based on the notion of \emph{proof search}, in which the execution of a program is defined as finding a proof of a particular proposition. (Again, this can be understood as \emph{program synthesis}, where the computer automatically creates a program which satisfies a particular constraint.) Unfortunately, proof search is undecidable for the full linear logic, so proofs will have to be written manually in some cases. And even in the cases where it is decidable, it can be a fairly expensive procedure which is arguably not viable for smart contracts due to the cost of running execution under consensus. As we will see however, there exist alternative execution strategies, which are remarkably well-suited for executing smart contracts.

\section{Motivation}
There are several reasons for why we think a logical language based on linear logic is a promising approach when it comes to smart contract programming.

\begin{itemize}
  \item Formal logics is an old and well-studied family of mathematical systems.
  \item Linear logic can be interpreted as a logic concerning itself with \emph{state} and \emph{scarce resources}, both of which are central to smart contracts.
  \item Proofs in linear logic can be given several different Curry-Howard correspondences, granting programmers the flexibility to write proofs/programs in several different programming languages, depending on which one is best suited for the task at hand. For example:
  \begin{itemize}
  		\item Terms in the $\pi$-calculus can be used. This is one of the most well-studied concurrent computational models. If blockchain sharding becomes reality, concurrency will almost certainly play a central role, thus making languages that can deal with it in a native and principled way ideal candidates for smart contract languages of the future.
  		\item In the same way, terms in the $\lambda$-calculus can be used, granting programmers a rather strict programming interface for when they actively don't want to consider the complexities of concurrent programs.
  \end{itemize}
\end{itemize}

\subsection*{My Motivation}
\TODO{!!!}

"Programming" is viewed as a process of Reason about Thoughts. The derived conclusion can be viewed as the final programm.

\subsection*{Unification}

\TODO{I want to make the point that logical programming > functional programming}

\TODO{Unification > pattern Matching}
\TODO{give example with $append([1], [2,3], X); X = [1,2,3] AND append([1], X, [1,2,3]); X = [2,3]$}

\TODO{backwards reasoning, bidirectional pattern matching. Which is the equivalent to the ability of asking ''what input should i make, to end up in a given state (e.g. invalid or indesirable)''}
\TODO{another example: ''X = 1'' and ''1 = X'' are the same statements - which is NOT a classical variable assignment. This allowes standard forward reasoning like with pattern matching but allowes backward reasoning + inconsistency checks}
\TODO{Another example is: ''X = Y; Y = 1'' with is not possible in imperative or functional programming since Y is not defined in the first statement. However ''X = Y; Y = 1; X = 2'' will fail due to an inconsistency of the system, because X is already ''equal to 1''}

Unification can be performed in quasi linear time\cite{Baader2001}.


\subsection{Requirements}

\begin{enumerate}
  \item[R01] Programming obvious applications has to be intuitive.
  \item[R02] Practically proving the correctness of programs must be possible.
  \item[R03] Execution of compiled programs must be relatively efficient.
  \item[R04] Programs have to be composable with other programs (i.e. programs are used to build larger programs, are used to build larger programs).
\end{enumerate}

\section{General idea}
On a high level, the idea is to view the blockchain as a multiset $\mathcal{B}$ of resources. A transaction consists of taking a subset $\Delta$ of those resources, using them to create a new resource $C$, and replacing $\Delta$ with $C$ in $\mathcal{B}$. More precisely, the resources in the blockchain $\mathcal{B}$ are linear logical propositions, and selecting a subset of them

\subsection{Unification}
During a blockchain transition step a Most General Unifier(MGU) $\theta$ has to be found. Initially $\theta$ contains the assignment of reserved variables, such as $Sender, Origin, BlockNumber, $ etc, which don't change during unification.

\begin{equation}
  \prftree[r]
  {$\mathcal{B}$-transition}
  {
    \prftree
    {
      \mathcal{D}
    } {
      \Delta\vdash C
    }
  }
  {\exists\theta. \Delta\theta \subseteq \mathcal{B}\theta}
  {\mathcal{B} \longrightarrow \mathcal{B} \setminus \Delta\theta \cup \{C\theta\}}
\end{equation}

\section{example programs}

Lets consider a simple token.


\section{Security}
We are interested to proof different properties about our written programs. E.g. that the total value of a system is not changed with different operations, this we want to illustrate on a case study - a coin exchange:
Let $d$ denote a dime, $n$ a nickel and $q$ a quarter together with the following persistent rules given:

\[\left.\\
\begin{array}{l}
r1: q\multimap d\otimes d\otimes n \\
r2: d\otimes d\otimes n\multimap q \\
r3: d\multimap n\otimes n \\
r4: n\otimes n\multimap d \\
\end{array}
\right\} R\]

Also we want to have a persistent notion of value:

\[ \left.
\begin{array}{l}
  v1: q\multimap value(25) \\
  v2: d\multimap value(10) \\
  v3: n\multimap value(5)
\end{array}
\right\} V \]

Additionally we can add up values:
\[ sum: value(X)\otimes value(Y)\multimap value(X + Y) \]

Now our task is to prove the stability of the system. This means that the total value don't change if a rule in R is applied. We can do this easily by proof expansion of every rule in $R$.
We want to illustrate this for $r1$, the other cases can be shown analogous:
\\
Without loss of generality let $\Delta = value(X), q$, we want to proof $\Delta \vdash value(X + 25)$ first by applying only Rules out of V + sum as seen in Proof \ref{pf1} and expand this proof by applying $r1$ first and then restricting ourselves with only using rules out of V, which can be seen in Proof \ref{pf2}. This proves that the application of rule $r1$ didn't change the total value.
With this we have shown that rules in R satisfy our stability criteria.
The proves are rather technical and included for the purpose to demonstrate, that those proves can be easily automatized.
\TODO{provided that the computation is deterministic and arbitrary values can't be derived. Find a way how to prove those properties automatically.}

\subsection*{Increment and safe transformation and simplification of programs}

Let $num$ be a binary numerical type given inductivelly by the following syntax:
\[ b :=\ 0\ |\ 1\ \]
\[ num :=\ b\ |\ num\ b \]
Let $inc$ be an two-ary increment predicate over num such that:

\[ inc(X, Y) \Leftrightarrow X + 1\ \%\ MAXINT = Y \]

We give the proof terms by subdeviding them into increment with overflow ($inc_o$) and a safe increment without overflow ($inc_s$):

\[ inc(X, Y) \equiv inc_o(X, Y) \oplus inc_s(X, Y)\]

Proof is given by the following rules for $inc_s$:

\begin{gather*}
  \prftree[r]{$i_{s0}$}{\vdash inc_s(0, 1)} \\
  \prftree[r]{$i_{sx0}$}{\Psi, X:num \vdash inc_s(X0, X1)} \\
  \prftree[r]{$i_{ss}$}{\Psi, X:num, Y:num \vdash inc_s(X,Y)}{inc_s(X1, Y0)}
\end{gather*}


and the following rules for $inc_o$:

\begin{gather*}
  \prftree[r]{$i_{o0}$}{\vdash inc_o(1,0)} \\
  \prftree[r]{$i_{os}$}{\Psi, X:num, Y:num \vdash inc_o(X,Y)}{inc_o(X1, Y0)}
\end{gather*}

Further let $div2$ be a two-ary operation over num, denoting a binary right shift operation/ division by two, given by the following rules:

\begin{gather*}
  \prftree[r]
    {$div2_0$}
    {\vdash div2(0, 0)}
    \\
  \prftree[r]
    {$div2_0$}
    {\vdash div2(1, 0)}\\
  \prftree[r]
    {$div2_{A0}$}
    {\Psi, A:b, B:b\vdash div2(AB, 0A)}
    \\
  \prftree[r]
    {$div2_{step}$}
    {\Psi, X:num, Y:num, A:b\vdash div2(XA, Y)}
    {\Psi, X:num, Y:num, A:b, B:b\vdash div2(XAB, YA)}
\end{gather*}

Now we can use both predicates to build Programms:

\begin{eqnarray}
  \vdash !div2(X, Y)\otimes !inc(Y, Z)
\end{eqnarray}

In this particular programm we first divide a number by two and then increment it. However, because we defined $inc$ with the help of two different predicates, the following transformation can be done safe and automatically:

\begin{eqnarray}
  \vdash !div2(X, Y) \otimes !(div_o(Y, Z)\oplus div_s(Y, Z))
\end{eqnarray}

stating that after we devide a number by two we either increment it with an overflow or without an overflow. But since after we devide a number by two no overflow on an increment can occur we should be able to abandon $inc_o$ and be able to prove the following program:

\begin{eqnarray}
  \vdash !div2(X, Y)\otimes !div_s(Y, Z)
\end{eqnarray}

Note that this transformation can be made automatically with the following provable rule:

\begin{eqnarray}
  !(A\otimes B) \vdash !A \otimes !B
\end{eqnarray}

Indeed we are able to prove this with structuran induction over X. Meaning we have the following rule for a given proposition P available (for expanded $b$):

\begin{equation}
  \prftree[r]
    {$induct_X$}
    {
      \prftree[noline]
      {
        \prftree[noline]
        {\Psi, \Gamma;\cdot\vdash P(0)}
        {\Psi, \Gamma;\cdot\vdash P(1)}
      }
      {
        \prftree[noline]
        {\Psi, X:num; \Gamma, P(X);\cdot\vdash P(X 0)}
        {\Psi, X:num; \Gamma, P(X);\cdot\vdash P(X 1)}
      }
    }
    {\Psi, X:num; \Gamma;\cdot\vdash P(X)}
\end{equation}

Let $P(x)$ be $div2(X, Y)\otimes inc(Y, Z)$, then we can prove the base cases ($P(0), P(1)$):

\subsubsection*{P(0):}
\begin{eqnarray}
  \prftree[r]
    {$\otimes$R}
    {
      \prftree[r] {$!R$}
      {
        \prftree[r] {$div2_{0}$}
        { \vdash div2(0, Y) }
      }
      { \vdash !div2(0, Y) }
    }
    {
      \prftree[r]{$!R$}
      {
        \prftree[r] {$i_{s0}$}
        { \vdash inc_s(Y, Z) }
      }
      { \vdash !inc_s(Y, Z) }
    }
    {\vdash !div2(0, Y)\otimes !inc_s(Y, Z) }
\end{eqnarray}
with $\theta = \{Y\mapsto 0, Z\mapsto 1\}$.


\subsubsection*{P(1):}
\begin{eqnarray}
  \prftree[r]
    {$\otimes$R}
    {
      \prftree[r]{$!R$}
      {
        \prftree[r] {$div2_{0}$}
        { \vdash div2(1, Y) }
      }
      { \vdash div2(1, Y) }
    }
    {
      \prftree[r]{$!R$}
      {
        \prftree[r] {$i_{s0}$}
        { \vdash inc_s(Y, Z) }
      }
      { \vdash !inc_s(Y, Z) }
    }
    {\vdash !div2(1, Y)\otimes !inc_s(Y, Z) }
\end{eqnarray}
with $\theta = \{Y\mapsto 0, Z\mapsto 1\}$.

the successor cases $P(x0)$:

\begin{eqnarray}
  \prftree[r]{$\otimes$L}
  {
    \prftree[r]{$\otimes$R}
    {
      \prftree[r]{$!R$}
      {
        \prftree[r] {$div2_{step}$}
        {
          \prftree[r] {$!L\circ$ copy}
          {
            \prftree[r] {id}
            { div2(X,Y)\vdash div2(X, Y) }
          }
          { !div2(X,Y)\vdash div2(X, Y) }
        }
        { !div2(X,Y)\vdash div2(X0, Y') }
      }
      { !div2(X,Y)\vdash !div2(X0, Y') }
    }
    {
      \prftree[r] {$induct_A$}
      {
        \mathcal{D}
      }
      { !inc_s(Y, Z)\vdash !inc_s(Y', Z') }
    }
    { !div2(X,Y), !inc_s(Y, Z)\vdash !div2(X0, Y')\otimes !inc_s(Y', Z') }
  }
  {!div2(X,Y)\otimes !inc_s(Y, Z)\vdash !div2(X0, Y')\otimes !inc_s(Y', Z')}
\end{eqnarray}
with $\mathcal{D} :=$
\begin{eqnarray}
  \prftree[r]{$induct_A$}
  {
    \prftree[r]{$!L$}
    {
      \prftree[r]{$!R$}
      {
        \prftree[r]{$i_{sx0}$}
        { \vdash inc_s(Y0, Z') }
      }
      { \vdash !inc_s(Y0, Z') }
    }
    { !inc_s(Y,Z)\vdash !inc_s(Y0, Z') }
  }
  {
    \prftree[r]{$!R$}
    {
      \prftree[r]{$!L\circ$ copy}
      {
        \prftree[r]{$i_{ss}$}
        {
          \prftree[r]{id}
          { inc_s(Y,Z)\vdash inc_s(Y, Z ) }
        }
        { inc_s(Y,Z)\vdash inc_s(Y1, Z') }
      }
      { !inc_s(Y,Z)\vdash inc_s(Y1, Z') }
    }
    { !inc_s(Y,Z)\vdash !inc_s(Y1, Z') }
  }
  {!inc_s(Y, Z)\vdash !inc_s(Y', Z')}
\end{eqnarray}
with $\theta = \{Y' \mapsto YA\}$\\
Note that in the left case of $\mathcal{D}$ - $Z' = Z1$ and in the left case $Z' = Y0$ and therefore not part of the most gerneral substitution.


the successor cases $P(x1)$:

\begin{eqnarray}
  \prftree[r]{$\otimes$L}
  {
    \prftree[r]{$\otimes$R}
    {
      \prftree[r]{$!R$}
      {
        \prftree[r]{$div2_{step}$}
        {
          \prftree[r]{$!L\circ$copy}
          {
            \prftree[r]{id}
            {
              div2(X,Y) \vdash div2(X, Y)
            }
          }
          {
            !div2(X,Y) \vdash div2(X, Y)
          }
        }
        {
          !div2(X,Y) \vdash div2(X1, Y')
        }
      }
      {
        !div2(X,Y) \vdash !div2(X1, Y')
      }
    }
    {
      \prftree[r]{$induct_A$}
      {\mathcal{D}}
      {
        !inc_s(Y,Z)\vdash !inc_s(Y', Z')
      }
    }
    { !div2(X,Y), !inc_s(Y,Z)\vdash !div2(X1, Y')\otimes !inc_s(Y', Z') }
  }
  {!div2(X,Y)\otimes !inc_s(Y,Z)\vdash !div2(X1, Y')\otimes !inc_s(Y', Z')}
\end{eqnarray}
with $\theta := \{Y' \mapsto YA\}$

On the other hand we won't be able to prove the overflow case because the base case for the induction will fail.
\[ \not\vdash div2(X,Y)\otimes inc_o(Y,Z) \]

By proving this, we have shown the general case: for all X we never encounter an overflow by computing Z.


\TODO{How can i do the reverse? Have a mathematical operation such as $mul2(X,Y)\otimes inc(Y,Z)$ and ask for the requirements for X to produce an overflow?}

\section{Efficiency}
As computation is done via proof-search and unification, executing programms tend to be slow in practice. Therefore we are interested in methods which reduces the computational overhead. One such method could be the recuction of proof search to the execution of a minimal set of evm instructions, which yield the same result. For this the equivalence of the proof search to the evm instructions has to be \textbf{proven}.

In order to achieve this, we will reconstruct the evm in linear logic and show for a proposition $A$, a proposition $B$ which represents an evm program and a substitution $\theta$:

\[ \Gamma; \cdot \vdash A(\bar X)\theta \Leftrightarrow \Gamma; \cdot \vdash B(\bar X)\theta \]

\TODO{Or is it $\Gamma; \cdot \vdash A(\bar X)\multimapboth B(\bar X)$?}

Meaning that the evm programm $B$ is behaviourally isomorphic to A.

\subsection*{EVM implementation}
For demonstration purposes we will just implement a basic stack machine here, which should be sufficient to illustrate the approach.

The EVM has a few distinct components, such as:

\begin{itemize}
  \item code
  \item pc
  \item stack
  \item ...
\end{itemize}


We can model the Stack Machine with the following predicates:

\begin{itemize}
  \item code(X,Y) - OpCode Y is on position X in the code.
  \item pc(X) - Program Counter is at position X
  \item stack(X,Y) - word Y is on position X on the stack
  \item ...
\end{itemize}


Additionally we create a few extra predicates which will help during the execution:
\begin{itemize}
  \item stackLength(X) - X is the current stack length
\end{itemize}

We now want to model the predicate $inc(X,Y) := X + 1 = Y$ in the evm with the following program: $PUSH\ X\ INC$\\ The result should then be found at stack height 0. We can retrieve it with:
\[ stack(0, X) \otimes pc(3) \multimap result(X) \]


Also initially we have the following predicates given:
$pc(0) \otimes stackLength(0)$

\subsection*{PUSH}

  $$ pc(X)\otimes code(X, PUSH) \otimes code(sX, V)\otimes stackLength(L)$$
  $$\multimap$$
  $$stack(L, V)\otimes pc(ssX)\otimes code(X, PUSH)\otimes code(sX, V)\otimes stackLength(sL)$$

This pushes the value V to the stack, increments the stack length and preserves the code.

\subsection*{INC}

Assume we have the following axiomatic rule for inc given:

\begin{itemize}
  \item inc(X, sX)
\end{itemize}

Thus:
  $$ inc(X, Y) \Leftrightarrow X + 1 = Y$$


So the following transition rule models our INC opCode:

  $$pc(X) \otimes code(X, INC) \otimes stack(L, V) \otimes stackLength(sL) \otimes inc(V, W)$$
  $$\multimap$$
  $$pc(sX) \otimes code(X, INC) \otimes stackLength(sL) \otimes stack(L, W)$$


\subsection*{proof}

Together we have to prove:
\[ inc(X,Y) \multimapboth (\mathcal{C}\otimes \mathcal{R}\otimes pc(0)\otimes stackLength(0) \multimap result(Y)) \]
with the code:
\[ \mathcal{C} := code(0, PUSH)\otimes code(1, X) \otimes code(2, INC) \]
and the termination condition:
\[ \mathcal{R} := pc(3)\otimes stackLength(1) \otimes stack(0, Z)\multimap result(Z)\]


\TODO{Make the actual proof, but the conjecture is that this is proovable.}

\TODO{How does this work with non-primitives, e.g. loops, math-function e.g. log, exp}

\section{formal verification of bytecode}
\TODO{explain the syntactic suggar notation: \$ - persist within a lolli, (omitting $\otimes$)}
\TODO{implicit: $\$code(PC, <opcode>)$}

Formal verificati of contracts on the opcode level is possible in Linear Logic. For this we will take the same aproach as described in Section EVM Implementation, but this time model teh actual EVM as described in \cite{Wood2014} rather then a POC stack machine, and try to prove the same properties about Simple Wallet Yoichi Hirai did in his ETH-Isabelle implementation\footnote{https://github.com/pirapira/eth-isabelle/blob/74aa293a16bd0b4d7449857e13b101d77d019185/example/SimpleWallet.thy}

Since the whole programm uses only a subset of opcodes we will only model those. Namely:

 STOP, EQ, ADDRESS, BALANCE, CALLER, SLOAD, JUMPI, GAS, JUMPDEST, DUP1, CALL

Further we assume a simplified EVM model in which we will need the following concepts:

\begin{itemize}
  \item address
  \item sender
  \item balance
  \item storage
  \item gas
  \item stack
  \item pc
  \item code
  \item sh
  \item memory
\end{itemize}

For now, we will ignore OutOfGas errors and will look further into it later.

\subsection*{Theorems}
Yoishi has Proven two main theorems about the contract:

\begin{enumerate}
  \item If a special address calls the contract, all balances of the contract is whisdrawn to this address
  \item If a non-special address calls the contract, the contract returns
\end{enumerate}

The special address is located at storage position 0.



\subsubsection*{PUSH1}

\begin{llstep}
  pc(PC)           &           & pc(PC + 2) \\
  gas(X + 3)       &           & gas(X) \\
  sh(SH)           & \multimap & sh(SH + 1) \\
  \$code(PC+1, V)  &           & \\
                   &           & stack(SH, V)
\end{llstep}


\subsubsection*{SLOAD}

\begin{llstep}
  pc(PC)            &           & pc(PC + 1) \\
  gas(X + 200)      &           & gas(X) \\
  \$sh(SH + 1)      & \multimap & \\
  stack(SH, KEY)    &           & stack(SH, V) \\
  \$storage(KEY, V) &           & \\
\end{llstep}


\subsubsection*{CALLER}

\begin{llstep}
  pc(PC)        &           & pc(PC + 1) \\
  gas(X + 2)    &           & gas(X) \\
  sh(SH)        & \multimap & sh(SH + 1) \\
  \$sender(A)   &           & \\
                &           & stack(SH, A)
\end{llstep}


\subsubsection*{ADDRESS}

\begin{llstep}
  pc(PC)        &           & pc(PC + 1) \\
  gas(X + 2)    &           & gas(X) \\
  sh(SH)        & \multimap & sh(SH + 1) \\
  \$address(A)  &           & \\
                &           & stack(SH, A)
\end{llstep}


\subsubsection*{EQ}

\begin{llstep}
  pc(PC)            &           & pc(PC + 1) \\
  gas(X + 3)        &           & gas(X) \\
  sh(SH + 2)        &           & sh(SH + 1) \\
  stack(SH, X)      & \multimap & \\
  stack(SH + 1, Y)  &           & \\
                    &           & (X=Y\multimap stack(SH, 1)\\
                    &           & \oplus X\neq Y\multimap stack(SH, 0))
\end{llstep}


\subsubsection*{JUMPI}

\begin{llstep}
  pc(PC)              &           & \\
  sh(SH + 2)          &           & sh(SH) \\
  gas(X + 10)         &           & gas(X) \\
  stack(SH, C)        & \multimap & \\
  stack(SH + 1, NPC)  &           & \\
                      &           & (C\neq 0\multimap pc(NPC)\\
                      &           & \oplus C=0 \multimap pc(PC + 1))
\end{llstep}


\subsubsection*{DUP1}

\begin{llstep}
  pc(PC)          &           & pc(PC + 1) \\
  gas(X + 3)      &           & gas(X) \\
  sh(SH + 1)      & \multimap & sh(SH + 2) \\
  \$stack(SH, V)  &           & \\
                  &           & stack(SH + 1, V)
\end{llstep}


\subsubsection*{BALANCE}

\begin{llstep}
  pc(PC)          &           & pc(PC + 1) \\
  gas(X + 400)    &           & gas(X) \\
  \$sh(SH + 1)    & \multimap & \\
  stack(SH, A)    &           & stack(SH, B) \\
  \$balance(A, B) &           &
\end{llstep}

\TODO{what happenes if balance predicate is not found for the account A? this should yield B being 0, how to model this without $\oplus$. Meanwhile pretend that all addresses are intitialized with 0}

\subsubsection*{GAS}

\begin{llstep}
  pc(PC)     &           & pc(PC + 1) \\
  gas(X + 2) &           & gas(X) \\
  sh(SH)     & \multimap & sh(SH + 1) \\
  gas(X)     &           & \\
             &           & stack(SH, X)
\end{llstep}

\subsubsection*{STOP}

\begin{llstep}
  pc(PC) & \multimap & pc(PC + 1) \\
         &           & STOP
\end{llstep}

\subsubsection*{JUMPDEST}

\begin{llstep}
  pc(PC)     & \multimap & pc(PC + 1) \\
  gas(X + 1) &           & gas(X)
\end{llstep}


\subsubsection*{CALL}

\begin{llstep}
  pc(PC) & & pc(PC + 1) \\
  sh(SH + 8) && sh(SH + 1) \\
  stack(SH)
\end{llstep}

\TODO{here a bunch of stuff happens, model it!}
\TODO{since in here we are just interested in the balance, maybe it is sufficient to just model the value transfear of call and keep everything else in place? Depending on how complex the full modell is}


\subsection*{Programm}

The program code:

\[
  \begin{array}{rcl}
    C & :=      & code(0, PUSH) \\
      & \otimes & code(1, 0) \\
      & \otimes & code(2, SLOAD) \\
      & \otimes & code(3, CALLER) \\
      & \otimes & code(4, EQ) \\
      & \otimes & code(5, PUSH) \\
      & \otimes & code(6, 9) \\
      & \otimes & code(7, JUMPI) \\
      & \otimes & code(8, STOP) \\
      & \otimes & code(9, JUMPDEST) \\
      & \otimes & code(10, PUSH) \\
      & \otimes & code(11, 0) \\
      & \otimes & code(12, DUP1) \\
      & \otimes & code(13, DUP1) \\
      & \otimes & code(14, DUP1) \\
      & \otimes & code(15, ADDRESS) \\
      & \otimes & code(16, BALANCE) \\
      & \otimes & code(17, CALLER) \\
      & \otimes & code(18, GAS) \\
      & \otimes & code(19, CALL) \\
      & \otimes & code(20, STOP)
  \end{array}
\]

\subsection*{Initial State}

\[
  \begin{array}{rcl}
    I&:=&pc(0)\\
    &\otimes&sh(0)\\
    &\otimes&gas(G)\\
    &\otimes&storage(0, W)\\
    &\otimes&address(A)\\
    &\otimes&balance(A, BA)\\
    &\otimes&balance(S, BS)\\
    &\otimes&sender(S)
  \end{array}
\]

\subsection*{Theorems}

\subsubsection*{Helpers}

For the theorems we additionally allow the weakening of the context, which is happening at the end to simplify the actual theorem:

\[
  I \multimap balance(A, BA) \otimes balance(S, BS)
\]

\[
  C \multimap 1
\]

\subsubsection*{Theorem 1.}
\[
  C \otimes I \otimes W = S \vdash balance(A, 0) \otimes balance(S, BS + BA) \otimes STOP
\]

\subsubsection*{step trough}

\[
  \begin{array}{rcl}
    &&pc(0)\\
    &\otimes&sh(0)\\
    &\otimes&gas(G)\\
    &\otimes&storage(0, W)\\
    &\otimes&address(A)\\
    &\otimes&balance(A, BA)\\
    &\otimes&balance(S, BS)\\
    &\otimes&sender(S) \\
    &\otimes& W = S
  \end{array}
\]
$\vdash$ \begin{flushright}(PUSH1)\end{flushright}

\[
  \begin{array}{rcl}
    &&pc(2)\\
    &\otimes&sh(1)\\
    &\otimes&gas(G' + 3)\\
    &\otimes&storage(0, W)\\
    &\otimes&address(A)\\
    &\otimes&balance(A, BA)\\
    &\otimes&balance(S, BS)\\
    &\otimes&sender(S)\\
    &\otimes& W = S\\
    &\otimes& stack(0, 0)
  \end{array}
\]

$\vdash$ \begin{flushright}(SLOAD)\end{flushright}

\[
  \begin{array}{rcl}
    &&pc(3)\\
    &\otimes&sh(1)\\
    &\otimes&gas(G' + 203)\\
    &\otimes&storage(0, W)\\
    &\otimes&address(A)\\
    &\otimes&balance(A, BA)\\
    &\otimes&balance(S, BS)\\
    &\otimes&sender(S)\\
    &\otimes& W = S\\
    &\otimes& stack(0, W)
  \end{array}
\]

$\vdash$ \begin{flushright}(CALLER)\end{flushright}

\[
  \begin{array}{rcl}
    &&pc(4)\\
    &\otimes&sh(2)\\
    &\otimes&gas(G' + 205)\\
    &\otimes&storage(0, W)\\
    &\otimes&address(A)\\
    &\otimes&balance(A, BA)\\
    &\otimes&balance(S, BS)\\
    &\otimes&sender(S)\\
    &\otimes& W = S\\
    &\otimes& stack(0, W)\\
    &\otimes& stack(1, S)
  \end{array}
\]

$\vdash$ \begin{flushright}(EQ)\end{flushright}

\[
  \begin{array}{rcl}
    &&pc(5)\\
    &\otimes&sh(1)\\
    &\otimes&gas(G' + 208)\\
    &\otimes&storage(0, W)\\
    &\otimes&address(A)\\
    &\otimes&balance(A, BA)\\
    &\otimes&balance(S, BS)\\
    &\otimes&sender(S)\\
    &\otimes& W = S\\
    &\otimes& stack(0, 1)
  \end{array}
\]

$\vdash$ \begin{flushright}(PUSH)\end{flushright}

\[
  \begin{array}{rcl}
    &&pc(7)\\
    &\otimes& sh(2)\\
    &\otimes& gas(G' + 211)\\
    &\otimes& storage(0, W)\\
    &\otimes& address(A)\\
    &\otimes& balance(A, BA)\\
    &\otimes& balance(S, BS)\\
    &\otimes& sender(S)\\
    &\otimes& W = S\\
    &\otimes& stack(0, 1)\\
    &\otimes& stack(1, 9)
  \end{array}
\]

$\vdash$ \begin{flushright}(JUMPI)\end{flushright}

\[
  \begin{array}{rcl}
    &&pc(9)\\
    &\otimes& sh(0)\\
    &\otimes& gas(G' + 221)\\
    &\otimes& storage(0, W)\\
    &\otimes& address(A)\\
    &\otimes& balance(A, BA)\\
    &\otimes& balance(S, BS)\\
    &\otimes& sender(S)\\
    &\otimes& W = S
  \end{array}
\]

$\vdash$ \begin{flushright}(JUMPDEST)\end{flushright}

\[
  \begin{array}{rcl}
    &&pc(10)\\
    &\otimes& sh(0)\\
    &\otimes& gas(G' + 222)\\
    &\otimes& storage(0, W)\\
    &\otimes& address(A)\\
    &\otimes& balance(A, BA)\\
    &\otimes& balance(S, BS)\\
    &\otimes& sender(S)\\
    &\otimes& W = S
  \end{array}
\]

$\vdash$ \begin{flushright}(PUSH)\end{flushright}

\[
  \begin{array}{rcl}
    &&pc(12)\\
    &\otimes& sh(1)\\
    &\otimes& gas(G' + 225)\\
    &\otimes& storage(0, W)\\
    &\otimes& address(A)\\
    &\otimes& balance(A, BA)\\
    &\otimes& balance(S, BS)\\
    &\otimes& sender(S)\\
    &\otimes& W = S\\
    &\otimes& stack(0, 0)
  \end{array}
\]

$\vdash$ \begin{flushright}(DUP1)\end{flushright}

\[
  \begin{array}{rcl}
    &&pc(13)\\
    &\otimes& sh(2)\\
    &\otimes& gas(G' + 228)\\
    &\otimes& storage(0, W)\\
    &\otimes& address(A)\\
    &\otimes& balance(A, BA)\\
    &\otimes& balance(S, BS)\\
    &\otimes& sender(S)\\
    &\otimes& W = S\\
    &\otimes& stack(0, 0)\\
    &\otimes& stack(1, 0)
  \end{array}
\]


$\vdash$ \begin{flushright}(DUP1)\end{flushright}

\[
  \begin{array}{rcl}
    &&pc(14)\\
    &\otimes& sh(3)\\
    &\otimes& gas(G' + 231)\\
    &\otimes& storage(0, W)\\
    &\otimes& address(A)\\
    &\otimes& balance(A, BA)\\
    &\otimes& balance(S, BS)\\
    &\otimes& sender(S)\\
    &\otimes& W = S\\
    &\otimes& stack(0, 0)\\
    &\otimes& stack(1, 0)\\
    &\otimes& stack(2, 0)
  \end{array}
\]


$\vdash$ \begin{flushright}(DUP1)\end{flushright}

\[
  \begin{array}{rcl}
    &&pc(15)\\
    &\otimes& sh(4)\\
    &\otimes& gas(G' + 234)\\
    &\otimes& storage(0, W)\\
    &\otimes& address(A)\\
    &\otimes& balance(A, BA)\\
    &\otimes& balance(S, BS)\\
    &\otimes& sender(S)\\
    &\otimes& W = S\\
    &\otimes& stack(0, 0)\\
    &\otimes& stack(1, 0)\\
    &\otimes& stack(2, 0)\\
    &\otimes& stack(3, 0)
  \end{array}
\]

$\vdash$ \begin{flushright}(ADDRESS)\end{flushright}

\[
  \begin{array}{rcl}
    &&pc(16)\\
    &\otimes& sh(5)\\
    &\otimes& gas(G' + 236)\\
    &\otimes& storage(0, W)\\
    &\otimes& address(A)\\
    &\otimes& balance(A, BA)\\
    &\otimes& balance(S, BS)\\
    &\otimes& sender(S)\\
    &\otimes& W = S\\
    &\otimes& stack(0, 0)\\
    &\otimes& stack(1, 0)\\
    &\otimes& stack(2, 0)\\
    &\otimes& stack(3, 0)\\
    &\otimes& stack(4, A)
  \end{array}
\]

$\vdash$ \begin{flushright}(BALANCE)\end{flushright}

\[
  \begin{array}{rcl}
    &&pc(17)\\
    &\otimes& sh(5)\\
    &\otimes& gas(G' + 636)\\
    &\otimes& storage(0, W)\\
    &\otimes& address(A)\\
    &\otimes& balance(A, BA)\\
    &\otimes& balance(S, BS)\\
    &\otimes& sender(S)\\
    &\otimes& W = S\\
    &\otimes& stack(0, 0)\\
    &\otimes& stack(1, 0)\\
    &\otimes& stack(2, 0)\\
    &\otimes& stack(3, 0)\\
    &\otimes& stack(4, BA)
  \end{array}
\]

$\vdash$ \begin{flushright}(CALLER)\end{flushright}

\[
  \begin{array}{rcl}
    &&pc(18)\\
    &\otimes& sh(6)\\
    &\otimes& gas(G' + 638)\\
    &\otimes& storage(0, W)\\
    &\otimes& address(A)\\
    &\otimes& balance(A, BA)\\
    &\otimes& balance(S, BS)\\
    &\otimes& sender(S)\\
    &\otimes& W = S\\
    &\otimes& stack(0, 0)\\
    &\otimes& stack(1, 0)\\
    &\otimes& stack(2, 0)\\
    &\otimes& stack(3, 0)\\
    &\otimes& stack(4, BA)\\
    &\otimes& stack(5, S)
  \end{array}
\]

$\vdash$ \begin{flushright}(GAS)\end{flushright}

\[
  \begin{array}{rcl}
    &&pc(19)\\
    &\otimes& sh(7)\\
    &\otimes& gas(G' + 640)\\
    &\otimes& storage(0, W)\\
    &\otimes& address(A)\\
    &\otimes& balance(A, BA)\\
    &\otimes& balance(S, BS)\\
    &\otimes& sender(S)\\
    &\otimes& W = S\\
    &\otimes& stack(0, 0)\\
    &\otimes& stack(1, 0)\\
    &\otimes& stack(2, 0)\\
    &\otimes& stack(3, 0)\\
    &\otimes& stack(4, BA)\\
    &\otimes& stack(5, S)\\
    &\otimes& stack(6, G' + 640)
  \end{array}
\]

$\vdash$ \begin{flushright}(CALL)\end{flushright}

\[
  \begin{array}{rcl}
    &&pc(20)\\
    &\otimes& sh(1)\\ % TODO - something
    &\otimes& gas(G' + 640)\\
    &\otimes& storage(0, W)\\
    &\otimes& address(A)\\
    &\otimes& balance(A, 0)\\
    &\otimes& balance(S, BS + BA)\\
    &\otimes& sender(S)\\
    &\otimes& W = S\\
    &\otimes& stack(0, 0)\\
  \end{array}
\]

$\vdash$ \begin{flushright}(STOP)\end{flushright}

\[
  \begin{array}{rcl}
    &&pc(21)\\
    &\otimes& sh(1)\\ % TODO - something
    &\otimes& gas(G' + 640)\\
    &\otimes& storage(0, W)\\
    &\otimes& address(A)\\
    &\otimes& balance(A, 0)\\
    &\otimes& balance(S, BS + BA)\\
    &\otimes& sender(S)\\
    &\otimes& W = S\\
    &\otimes& stack(0, 0)\\
    &\otimes& STOP
  \end{array}
\]
Now we optimize based on our helpers:

\[
  \begin{array}{rcl}
    && balance(A, 0)\\
    &\otimes& balance(S, BS + BA)\\
    &\otimes& STOP
  \end{array}
\]




\subsubsection*{Theorem 2.}
\[
  C \otimes I \otimes W \neq S \vdash balance(A, BA) \otimes balance(S, BS) \otimes STOP
\]




\section{further considerations}

\begin{itemize}
  \item case study: quicksort
  \begin{itemize}
    \item implement
    \item prove runntime is $O(n*log(n))$
    \item prove list is fully sorted after the "algorithm" has finished
    \item abstract away $<$ relation: e.g. forward and backward sorting
    \item explore actual runntime cost and explore optimization-techniques for proof-search based on custom heuristics
  \end{itemize}
\end{itemize}



\section{Appendix}

\begin{figure*}
\[
  \prftree[r] {$\multimap$L}
  {
    \prftree[r] {$\otimes$L}
    {
      \prftree[r] {$copy_{sum}$}
      {
        \prftree[r] {$\multimap$L}
        {
          \prftree[r] {$id(\sigma)$}
          {
            value(X)\otimes value(25)\vdash value(Y)\otimes value(Y')
          }
        }
        {
          \prftree[r] {$id(\sigma)$}
          {
            value(Y + Y')\vdash value(X + 25)
          }
        }
        {
          value(X)\otimes value(25), value(Y)\otimes value(Y')\multimap value(Y + Y')\vdash value(X + 25)
        }
      }
      { value(X)\otimes value(25)\vdash value(X + 25) }
    }
    { value(X),value(25)\vdash value(X + 25) }
  }
  {
    \prftree[r] {id}
    { q\vdash q }
  }
  { \prftree[r] {$copy_{v1}$}
    {value(X),q,q\multimap value(25)\vdash value(X + 25)}
    {value(X),q\vdash value(X + 25)}
  }
\]
$\sigma = \{\frac{X}{Y}, \frac{25}{Y'}\}$
\caption{Proof using only rules from $V$}
\label{pf1}
\end{figure*}

\begin{figure*}
  \[
    \prftree[r] {$copy_{r1}$} {
      \prftree[r] {$\multimap$L}
      {
        \prftree[r] {id} { q\vdash q }
      }
      {
        \prftree[r] {$\otimes$L}
        {
          \prftree[r] {$\otimes$L}
          {
            \prftree[r] {$copy_{v2}$}
            {
              \prftree[r] {$\multimap$L}
              {
                \prftree[r]{$\otimes$L}
                {
                  \prftree[r]{$copy_{sum}$}
                  {
                    \prftree[r] {$\multimap$L}
                    {
                      \prftree[r] {$copy_{v2}$}
                      {
                        \prftree[r] {$\multimap$L}
                        {
                          \prftree[r] {id} {d\vdash d}
                        }
                        {
                          \prftree[r] {$\otimes$L}
                          {
                            \prftree[r] {$copy_{sum}$}
                            {
                              \mathcal{D}
                            }
                            {
                              value(Y + Y')\otimes value(10), n \vdash value(X + 25)
                            }
                          }
                          {
                          value(Y + Y'), value(10), n \vdash value(X + 25)
                          }
                        }
                        {
                          value(Y + Y'), d, d\multimap value(10), n \vdash value(X + 25)
                        }
                      }
                      {
                      value(Y + Y'), d, n \vdash value(X + 25)
                      }
                    }
                    {
                      \prftree[r] {$id(\sigma)$}
                      {
                        value(X)\otimes value(10) \vdash value(Y)\otimes value(Y')
                      }
                    }
                    {
                      value(X)\otimes value(10), value(Y)\otimes value(Y')\multimap value(Y + Y'), d, n \vdash value(X + 25)
                    }
                  }
                  {
                    value(X)\otimes value(10), d, n \vdash value(X + 25)
                  }
                }
                {
                  value(X), d, n, value(10)\vdash value(X + 25)
                }
              }
              {
                \prftree[r]{id} {d\vdash d}
              }
              {
                value(X), d, d, n, d\multimap value(10)\vdash value(X + 25)
              }
            }
            {
              value(X), d, d, n\vdash value(X + 25)
            }
          }
          {
            value(X), d, d\otimes n\vdash value(X + 25)
          }
        }
        {
          value(X), d\otimes d\otimes n\vdash value(X + 25)
        }
      }
      {
        value(X), q, q\multimap d\otimes d\otimes n\vdash value(X + 25)
      }
    } {
      value(X), q \vdash value(X + 25)
    }
  \]
With $\mathcal{D} :=$
  \[
  \prftree[r] {$\multimap$L}
  {
      \prftree[r] {$id(\sigma)$}
      {
        value(Y + Y')\otimes value(10)\vdash value(Z)\otimes value(Z')
      }
    }
    {
      \prftree[r] {$copy_{v3}$}
      {
        \prftree[r] {$\multimap$L}
        {
          \prftree[r] {$\otimes$L}
          {
            \mathcal{E}
          }
          {
            value(Z + Z'), value(5) \vdash value(X + 25)
          }
        }
        {
          \prftree[r] {id} {n\vdash n}
        }
        {
          value(Z + Z'), n, n\multimap value(5) \vdash value(X + 25)
        }
      }
      {
        value(Z + Z'), n \vdash value(X + 25)
      }
    }
    {
      value(Y + Y')\otimes value(10), value(Z)\otimes value(Z')\multimap value(Z + Z'), n \vdash value(X + 25)
    }
  \]
  With $\mathcal{E} :=$
  \[
  \prftree[r] {$copy_{sum}$}
  {
    \prftree[r] {$\multimap$L}
    {
      \prftree[r] {$id(\sigma)$}
      {
        value(Z + Z')\otimes value(5)\vdash value(Q)\otimes value(Q')
      }
    }
    {
      \prftree[r] {$id(\sigma)$}
      {
        value(Q + Q') \vdash value(X + 25)
      }
    }
    {
      value(Z + Z')\otimes value(5), value(Q)\otimes value(Q')\multimap value(Q + Q') \vdash value(X + 25)
    }
  }
  {
    value(Z + Z')\otimes value(5) \vdash value(X + 25)
  }
  \]
  With $\sigma := \{\frac{X}{Y}, \frac{10}{Y'}, \frac{X + 10}{Z}, \frac{10}{Z'}, \frac{X + 10 + 10}{Q}, \frac{5}{Q'}\}$
  \caption{Proof using rule $r1$ and only rules from $V$}
  \label{pf2}
\end{figure*}

\bibliography{library}
\bibliographystyle{ieeetr}


\end{document}
